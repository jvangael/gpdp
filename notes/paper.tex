%%%%%%%%%%%%%%%%%%%%%%%%%%%%%%%%%%%%%%%%%%%%%%%%%%%%%%%%%%%%%%%%%%
%%%%%%%% ICML 2012 EXAMPLE LATEX SUBMISSION FILE %%%%%%%%%%%%%%%%%
%%%%%%%%%%%%%%%%%%%%%%%%%%%%%%%%%%%%%%%%%%%%%%%%%%%%%%%%%%%%%%%%%%

% Use the following line _only_ if you're still using LaTeX 2.09.
%\documentstyle[icml2012,epsf,natbib]{article}
% If you rely on Latex2e packages, like most moden people use this:
\documentclass{article}

\usepackage{graphicx}
\usepackage{subfigure} 
\usepackage{natbib}
\usepackage{algorithm}
\usepackage{algorithmic}
\usepackage{hyperref}
\usepackage{amssymb,amsmath,bm}

% Packages hyperref and algorithmic misbehave sometimes.  We can fix
% this with the following command.
\newcommand{\theHalgorithm}{\arabic{algorithm}}

\newcommand{\GP}[2]{\mathcal{GP}\left({#1},{#2}\right)}
\newcommand{\DP}[2]{\mathcal{DP}\left({#1},{#2}\right)}

% Employ the following version of the ``usepackage'' statement for
% submitting the draft version of the paper for review.  This will set
% the note in the first column to ``Under review.  Do not distribute.''
\usepackage[accepted]{icml2012} 
% Employ this version of the ``usepackage'' statement after the paper has
% been accepted, when creating the final version.  This will set the
% note in the first column to ``Appearing in''
% \usepackage[accepted]{icml2012}


% The \icmltitle you define below is probably too long as a header.
% Therefore, a short form for the running title is supplied here:
\icmltitlerunning{Nonparametric Noise Models for the Gaussian Process}

\begin{document} 

\twocolumn[
\icmltitle{Nonparametric Noise models for the Gaussian Process}

% It is OKAY to include author information, even for blind
% submissions: the style file will automatically remove it for you
% unless you've provided the [accepted] option to the icml2012
% package.
\icmlauthor{Ulrich Paquet}{ulripa@microsoft.com}
\icmladdress{Microsoft Research, Cambridge}
\icmlauthor{Jurgen Van Gael}{jurgen@rangespan.com}
\icmladdress{Rangespan Ltd., B131 MacMillan House, Paddington Station, London W2 1FT, UK}

% You may provide any keywords that you 
% find helpful for describing your paper; these are used to populate 
% the "keywords" metadata in the PDF but will not be shown in the document
\icmlkeywords{boring formatting information, machine learning, ICML}

\vskip 0.3in
]

\begin{abstract} 
Notes while developing a Gaussian Process with nonparametric noise model.
\end{abstract} 

\section{Introduction}
\label{intro}

It is sometimes really hard to tell what noise model we want to use for a GP. A specific example is quantile regression for product demand forecasting. We've got an underlying trend, perhaps with cyclical component, which we can easily model with a GP by encoding prior knowledge in the covariance matrix. Unfortunately sales data might have spike and other irregularities which make choosing a noise model quite tricky. One option is to use a robust noise model like student-t or Laplace. In this work, we learn a noise model by using a non-parametric mixture of Gaussians.



\section{Model}
\label{model}

Imagine we have a time series with observations $y_t$ at times $x_t$ with $t \in [0,T]$. We model this data by assuming a latent Gaussian process
\begin{equation}
\bm{f} \sim \GP{0}{\bm{K}(\bm{x},\bm{x})}
\end{equation}
We model the noise as a non-parametric mixture model using the Dirichlet process. Let
\begin{equation}
\bm{G} \sim \DP{\alpha}{H}
\end{equation}
be a Dirichlet process with concentration parameter $\alpha$ and base measure $H$. For each time $t$ we introduce a noise variable $\epsilon_t \sim G$. We then model the observation $y_t = \bm{f}(x_t) + \epsilon_t$.



\section{Inference}
\label{inference}

We can perform inference in this model using a collapsed Gibbs sampler. In order to work with the CRP representation of the Dirichlet process we introduce a new variable $z_t$ which will represent CRP partition that datapoint $t$ belongs to. For each CRP partition $n$ we represent the cluster parameters using $\theta_n$.

\begin{algorithm}[tb]
   \caption{Collapsed Gibbs Sampling}
   \label{alg:cgibbs}
\begin{algorithmic}
   \STATE {\bfseries Input:} data $\bm{x}, \bm{y}$ and kernel $\bm{K}(\bm{x},\bm{x})$
   \STATE {\bfseries Initialisation:} $z_t \sim \textrm{CRP}(\alpha)$, $\theta_n \sim H$
   \REPEAT
   \STATE Sample $z_t | \bm{K}, \bm{y}, \bm{z}_{\lnot t}, \bm{\theta}$
   \STATE Sample $\theta_n | \bm{x}, \bm{y}, \bm{\theta}$
   \UNTIL convergence
\end{algorithmic}
\end{algorithm}

In algorithm~\ref{alg:cgibbs} we integrate out the Gaussian process $\bm{f}$. In what follows we derive the resampling steps for $z_t$ and $\theta_n$.

\paragraph{Sampling $z_t$} The conditional distribution of $z_t$ can be written as follows
\begin{eqnarray*}
p(z_t | \bm{K}, \bm{y}, \bm{z}_{\lnot t}, \bm{\theta}) & \propto & \int p(\bm{y} | \bm{f}, \bm{z}_{\lnot t}, z_t, \bm{\theta} ) p(\bm{f} | \bm{K}) d \bm{f}, \\
 & = & \int p(y_t | f_t, \theta_{z_t} ) p(f_t | \bm{y}, \bm{z}_{\lnot t}, \bm{\theta}, \bm{K}) df_t.
\end{eqnarray*}

The key bit is that $p(f_t | \bm{y}, \bm{z}_{\lnot t}, \bm{\theta}, \bm{K})$ is a Gaussian process where every datapoint contributes noise that is dependent on the CRP partition it belongs to.

\begin{equation}
\begin{bmatrix}
\bm{y} \\ f_t
\end{bmatrix}
\sim \mathcal{N}\left(\bm{0},
\begin{bmatrix}
\bm{K}(\bm{x}, \bm{x}) + \bm{\Sigma}& \bm{K}(\bm{x},\bm{x_t}) \\
\bm{K}(\bm{x_t}, \bm{x}) & \bm{K}(\bm{x_t},\bm{x_t})
\end{bmatrix}
\right)
\end{equation}

Where $\bm{\Sigma}$ is a diagonal matrix with on the diagonal $\bm{\Sigma}_{ii} = \theta_{z_{ii}}$.

We know that \cite{gpml} $p(\bm{f} | \bm{K})$ can be represented as a multivariate Gaussian distribution. 

%  & = & \int p(y_t | f_t, \theta_{z_t} ) p(f_t | \bm{f}, \bm{y}, \bm{z}_{\lnot t}, z_t, \bm{\theta}) p(\bm{y}_{\lnot t} | \bm{f)_{\lnot t},  \bm{z}_{\lnot t}, \bm{\theta}) p(\bm{f} | \bm{K}) d \bm{f}, \\

\paragraph{Sampling $\theta_n$}






\subsection{Software and Data}


% Acknowledgements should only appear in the accepted version. 
\section*{Acknowledgments} 

\bibliography{paper}
\bibliographystyle{icml2012}

\end{document}

